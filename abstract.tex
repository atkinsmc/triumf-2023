\documentclass[a4paper]{article}

\title{\textit{Ab initio} calculation of the  $^3$He$(\alpha,\gamma)^7$Be astrophysical $S$ factor}
\author{M.C. Atkinson, K. Kravvaris, S. Quaglioni, G. Hupin, P. Navr\'{a}til}

\begin{document}

\maketitle

\section{Abstract}


 %The $^3$He$(\alpha,\gamma)^7$Be radiative-capture reaction rates between 20 and 500 keV are essential in understanding the primordial $^7$Li abundance in the universe. The Coulomb repulsion between the fusing nuclei suppresses the capture cross section at these low energies, making it difficult to measure directly. Theoretical calculations are needed to guide the extrapolation to the solar energies of interest.  To this end, I will present NCSMC calculations of the $^3$He$(\alpha,\gamma)^7$Be reaction within the no-core shell model with continuum starting from two- and three-nucleon chiral interactions. To demonstrate that the NCSMC provides an accurate $S$ factor, I will also compare NCSMC $^{3}$He + $^{4}$He elastic-scattering cross sections with those recently measured by the SONIK collaboration.

The $^3$He$(\alpha,\gamma)^7$Be reaction is an important part of ongoing processes occuring in all stars like our very own sun. In the fusion reaction
network of the sun, the $^3$He$(\alpha,\gamma)^7$Be reaction is key to determining the $^7$Be and $^8$B neutrino fluxes resulting from the pp-II chain
. In standard solar model (SSM) predictions of these neutrino fluxes, the low-energy $^3$He$(\alpha,\gamma)^7$Be $S$ factor, $S_{34}(E)$, is the
largest source of uncertainty from nuclear input. The SSM uses $S_{34}(E)$ near the Gamow peak energy, roughly 18 keV, which cannot be
experimentally measured since the Coulomb force between $^3$He and $^4$He suppresses the fusion reaction at such low energies. Theoretical calculations are needed to guide the extrapolation to the solar energies of interest.  To this end, I will present \textit{ab initio} calculations of the $^3$He$(\alpha,\gamma)^7$Be reaction using the no-core shell model with continuum starting from two- and three-nucleon chiral interactions. To demonstrate that the NCSMC provides an accurate $S$ factor, I will also compare NCSMC $^{3}$He + $^{4}$He elastic-scattering cross sections with those recently measured by the SONIK collaboration. 

%Additionally, I will compare the NCSMC $^{3}$He + $^{4}$He elastic scattering cross sections with those recently measured by the SONIK collaboration. The good agreement with the SONIK data verifies that the NCSMC scattering states are providing 
 %The NCSMC scattering states relevant for the capture calculation show good agreement with the recently-acquired SONIK $^{3}$He + $^{4}$He elastic scattering cross section data. 

 %are verified by calculating $^{3}$He + $^{4}$He elastic scattering cross sections 

This work was performed under the auspices of the U.S. Department of Energy by Lawrence Livermore National Laboratory under contract DE-AC52-07NA27344. Lawrence Livermore National Security, LLC.

\end{document}


